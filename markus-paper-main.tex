%%%%%%%%%%%%%%%%%%%%%%%%%%% asme2e.tex %%%%%%%%%%%%%%%%%%%%%%%%%%%%%%%
% Template for producing ASME-format articles using LaTeX            %
% Written by   Harry H. Cheng                                        %
%              Integration Engineering Laboratory                    %
%              Department of Mechanical and Aeronautical Engineering %
%              University of California                              %
%              Davis, CA 95616                                       %
%              Tel: (530) 752-5020 (office)                          %
%                   (530) 752-1028 (lab)                             %
%              Fax: (530) 752-4158                                   %
%              Email: hhcheng@ucdavis.edu                            %
%              WWW:   http://iel.ucdavis.edu/people/cheng.html       %
%              May 7, 1994                                           %
% Modified: February 16, 2001 by Harry H. Cheng                      %
% Modified: January  01, 2003 by Geoffrey R. Shiflett                %
% Use at your own risk, send complaints to /dev/null                 %
%%%%%%%%%%%%%%%%%%%%%%%%%%%%%%%%%%%%%%%%%%%%%%%%%%%%%%%%%%%%%%%%%%%%%%

%%% use twocolumn and 10pt options with the asme2e format
\documentclass[twocolumn,10pt]{asme2e}
\usepackage[utf8]{inputenc}
%% The class has several options
%  onecolumn/twocolumn - format for one or two columns per page
%  10pt/11pt/12pt - use 10, 11, or 12 point font
%  oneside/twoside - format for oneside/twosided printing
%  final/draft - format for final/draft copy
%  cleanfoot - take out copyright info in footer leave page number
%  cleanhead - take out the conference banner on the title page
%  titlepage/notitlepage - put in titlepage or leave out titlepage
%  
%% The default is oneside, onecolumn, 10pt, final

%%% Replace here with information related to your conference
\confshortname{ESDA 2012}
\conffullname{the ASME 2012 11th Biennial Conference On Engineering Systems Design \\ And Analysis}

%%%%% for date in a single month, use
\confdate{2-4}
\confmonth{July}
%%%%% for date across two months, use
%\confdate{August 30-September 2}
\confyear{2012}
\confcity{Nantes}
\confcountry{France}

%%% Replace DETC2010/MECH-12345 with the number supplied to you 
%%% by ASME for your paper.
\papernum{ESDA2012-82141}

%%% You need to remove 'DRAFT: ' in the title for the final submitted version.
\title{MARKUS, AN OPEN-SOURCE WEB APPLICATION TO ANNOTATE STUDENT PAPERS ON-LINE}

%%% first author
\author{Karen Reid
    \affiliation{
	University of Toronto\\
	40 St. George St.\\
	Toronto, Ontario M5S 2E4\\ 
	Canada\\
    Email: reid@cs.toronto.edu
    }	
}

\author{Mike Conley
    \affiliation{
	Mozilla Corp.\\
	Canada\\
    Email: mike.d.conley@gmail.com
    }	
}

\author{Severin Gehwolf
    \affiliation{
	University of Toronto\\
	40 St. George St.\\
	Toronto, Ontario M5S 2E4\\ 
	Canada\\
    Email: severin.gehwolf@utoronto.ca
    }	
}

\author{Morgan Magnin \\
       {\tensfb Guillaume Moreau}   
    \affiliation{
	LUNAM Universit\'{e} \\
	Ecole Centrale de Nantes\\
	1 rue de la No\"{e}, BP 92101\\
	44321 Nantes Cedex 3\\
	France\\
    Email: morgan.magnin@ec-nantes.fr \\
    guillaume.moreau@ec-nantes.fr
    }	
}


\author{Nelle Varoquaux
    \affiliation{
	Affiliation ?\\
    Email: nelle.varoquaux@gmail.com
    }	
}

\author{Benjamin Vialle
    \affiliation{
	Mobile Devices Ingenierie\\
	100, avenue de Stalingrad\\
	94800 Villejuif\\
		France\\
    Email: benjaminvialle@gmail.com
    }	
}

\begin{document}

\maketitle    

%%%%%%%%%%%%%%%%%%%%%%%%%%%%%%%%%%%%%%%%%%%%%%%%%%%%%%%%%%%%%%%%%%%%%%
\begin{abstract}
{\it A critical component of the learning process lies in the feedback that students receive on their work that validates their progress, identifies flaws in their thinking, and identifies skills that still need to be learned. Many higher-education institutions have developed an active pedagogy that gives students opportunities for different forms of assessment and feedback. This means that students have numerous lab exercises, assignments, and projects. Both instructors and students thus require effective tools to efficiently manage the submission, assessment, and individualized feedback of students' work. The open-source web application MarkUs aims at meeting these needs: it facilitates the submission and assessment of students? work. Students directly submit their work using MarkUs, rather than printing it, or sending it by email. The instructors or teaching assistants use MarkUs's interface to view the students' work, annotate it, and fill in a marking rubric. Students use the same interface to read the annotations and learn from the assessment. Managing the students' submissions and the instructors assessments within a single online system, has led to several positive pedagogical outcomes: the number of late submissions has decreased, the assessment time has been drastically reduced, students can access their results and read the instructor's feedback immediately after the grading process is completed. Using MarkUs has also significantly reduced the time that instructors spend collecting assignments, creating the marking schemes, passing them on to graders, handling special cases, and returning work to the students.

MarkUs was created and developed at University of Toronto (UofT), Canada. Since summer 2009, students at École Centrale de Nantes (ECN) have joined the MarkUs development team. In use at UofT, University of Waterloo (Canada) and ECN, MarkUs supports the assessment of several thousand students.

In this paper, we introduce MarkUs' features, and illustrate their benefits for higher education through our own teaching experiences and that of our colleagues. We also describe an important benefit of the fact that the tool itself is open-source. MarkUs has been developed entirely by students giving them a valuable learning opportunity as they work on a large software system that real users depend on. Virtuous circles indeed arise, with former users of MarkUs becoming developers and then supervisors of further development.

We will conclude by drawing perspectives about forthcoming features and use, both technically and pedagogically.}
\end{abstract}

%%%%%%%%%%%%%%%%%%%%%%%%%%%%%%%%%%%%%%%%%%%%%%%%%%%%%%%%%%%%%%%%%%%%%%
%\begin{nomenclature}
%\entry{A}{You may include nomenclature here.}
%\entry{$\alpha$}{There are two arguments for each entry of the nomemclature environment, the symbol and the definition.}
%\end{nomenclature}

%The spacing between abstract and the text heading is two line spaces.  The primary text heading is  boldface in all capitals, flushed left with the left margin.  The spacing between the  text and the heading is also two line spaces.

%%%%%%%%%%%%%%%%%%%%%%%%%%%%%%%%%%%%%%%%%%%%%%%%%%%%%%%%%%%%%%%%%%%%%%
\section*{INTRODUCTION}

% The main idea : Why is assessing student papers so important, why there are so many difficulties when many students and teachers are involved, how every one of us previously tried to manage with these difficulties, why a tool like MarkUs was needed

One critical pedagogical outcome lies in the optimization of the assessment of students' papers (project reports, homework, exams). Various approaches can be adopted to tackle this issue. One of our previous attempts was to simplify the management and evaluation of work submitted by students through the use of Tablet PCs \cite{magnin-tice-2010}. If the annotation work is facilitated, the management of students' papers still remains a hot topic: files have to be received by e-mail, temporary stored on the teacher's computer then returned corrected to the sender(s), etc. This method did not completely satisfy us, thus motivating us to turn to a new relevant and effective tool for the assessment of students' work: MarkUs \cite{markus}.

MarkUs takes into account the needs expressed by teachers and students. Thanks to a single web-application, it offers a wide range of useful features: 
\begin{itemize}
\item All documents submitted by students are centralized (no need to manually manage emails sent by students, no difficulties neither with inappropriate compressed file archives nor with enclosed files that are too heavy, no bad interaction with email spam filters, etc.. ) and versionned (successive versions of each file are stored, so that teachers can track the progress of students over time \cite{Reid05learningby}).
\item Annotations "types" may be proposed by default, based on the most common errors encountered in previous years;
\item The teacher can of course add his own comments, save some as "the most frequent comments" and reuse them in other assessments; 
\item The software manages the deadlines that students have to respect to submit their work. It is possible to specify additional features, like "grace period credits" or "penalty formulas";
\item Etc.
\end{itemize}

Students do not need anymore to print their code or to send it by mail to their teachers : all these steps are managed via MarkUs, which recreates the ease and flexibility of grading assignments with pen on paper. 

MarkUs was created at University of Toronto (UoT), Canada. Since Summer 2009, students at Ecole Centrale de Nantes (ECN) have joined the MarkUs development team. Already in use at UoT and University of Waterloo (Canada), MarkUs has been deployed at ECN in September 2010. In this paper, we present Markus' features, and illustrate their potential assets for higher education through our own feedback. We will not only provide an overview of current uses and future but also put the emphasis on the benefit taken from the fact that such a tool is open-source. 



%%%%%%%%%%%%%%%%%%%%%%%%%%%%%%%%%%%%%%%%%%%%%%%%%%%%%%%%%%%%%%%%%%%%%%
\section*{THE MARKUS TOOL}

% Presentation of the history, then features, of MarkUs

% If the heading should run into more than one line, the run-over is flush left.

%%%%%%%%%%%%%%%%%%%%%%%%%%%%%%%%%%%%%%%%%%%%%%%%%%%%%%%%%%%%%%%%%%%%%%
\subsection*{History}

%%% TO BE CORRECTED - start

%2006: OLM, then MarkUs
MarkUs was initially called OLM, for Online Marking Tool. Started in 2005, the
Online Marking tool was a web application motivated by improving the
effectiveness and efficiency of marking lower year Computer Science
assignments. It displayed all the needed information to mark an assignment:
The content, comments, marking scheme and more, all in one place. It also
supports easy migration by importing and exporting data in XML format.
Developed from 2005 to 2007 in Python using the TurboGears framework, 
the source code was not easy to understand and too messy to be correctly maintained.
% Perhaps a link to https://stanley.cdf.toronto.edu/drproject/csc49x/olm

Karen Reid, with the team agreement, decided to start a rewriting in Ruby using the Ruby on Rails framework in 2008. Called Markus. One year later, all functionality of OLM were implemented in Markus.

%%% TO BE CORRECTED - end

\subsection*{Free software philosophy}
Markus is a free software, i.e. a software whose use, study, editing and duplication for distribution are permitted, technically and legally, to guarantee a wide range of freedoms to the end-user. Licensed under MIT, it is developed mainly by Canadian and French students.
Presented at several conferences, Markus receives various outside contributions. Students and teachers from other universities send patches that are integrated with the main branch of development. The presentations in conference can also expand the range of returns on software, and identify additional needs.

One of the major added value of Markus is the integration of many components that are great successes of the free software community: version control module to collect the student code, use of a well-known javascript library to annotate images and PDF documents, integration of annotations in LaTeX and MathML to adapt the tool to the research environment. 

\subsection*{Community}
Markus is the subject of ongoing developments. Each semester, a dozen of new students participate in the implementation of its features. They are co-supervised by teachers and technical mentors (that often are former students who continue to participate in the evolution of the software).

Since 2008, more than 45 undergraduate students have participated in the development of MarkUs; some as full-time summer interns, but most working part time on MarkUs as a project course. The fact that we have have uncovered so few major bugs, and that MarkUs has been so well-received by instructors is a testament to the high quality work of these students. 

As the development of Markus takes profit from the dynamics of project-based learning, we have been able to establish virtuous circles: students, who were former Markus users, are invited to become contributors, and then act as technical mentors for new projects once they have been graduated. This positive cycle has been extensively described in \cite{magnin-qpes-2011}.

\subsection*{Features}
%Global presentation of the main MarkUs features

As a web-application, Markus is cross-platform (Windows, Mac OS X, Linux,
etc.). As MarkUs can be used through any web browser, it is therefore
well-suited to nomadism situations that students and faculty members may
encounter in their daily life.

%%% TO BE MODIFIED - beginning
Through the configuration of an assignment, intructors can enforce submission
deadlines, marking schemes, automatic penalties, and group and grader
management (max and min number of students per groups, who grades which
student group or what part of the assignment).

Once the assignment is properly set up, students can start forming groups by
inviting other students; accepting or rejecting invitations. Once group
requirements are met, they can start submitting their source code, diagrams,
pictures and schemas on MarkUs. In order to handle automatic version control,
a subversion repository is plugged in the backend of the webfile upload. For
more advanced courses, direct command line access to the repository is
possible.

Once the due date is passed, graders can easily annotate students' code,
diagrams and pictures. Markus allows instructors to create annotation
categories and load default annotations. Hence, graders do not need to retype
commonly used comments.

%%% TO BE MODIFIED - end

%%% Add one screenshot of MarkUs, maybe about annotation of code

%%%%%%%%%%%%%%%%%%%%%%%%%%%%%%%%%%%%%%%%%%%%%%%%%%%%%%%%%%%%%%%%%%%%%%
\section*{THE DEPLOYMENT OF MARKUS}

MarkUs has been used in Computer Science related courses since 2007 in Canada and since 2010 in France. In this section, 

The deployment in Canada, in France. How is it used? How teachers get used to the tool? How do they use it?

\subsection*{An overview of the use of MarkUs in Higher Education}

Markus is currently used at the University of Toronto, University of Waterloo and Ecole Centrale de Nantes: 
\begin{itemize}
\item In University of Toronto, ... %to be filled with info about the number of students and courses impacted
\item In University of Waterloo, ... %to be filled with info about the number of students and courses impacted
\item In Ecole Centrale de Nantes, MarkUs has been used since 2010. It impacts both undergraduate and master students, in Computer Science related courses like: algorithmics, C and Java programming, databases, modeling languages (UML). Every year, that is an average of 800 students and 20 teachers that use MarkUs. %to be completed
\end{itemize}

\subsection*{How teachers are accompanied to use the tool? How they use it?}

MarkUs is easy to handle, thus its use does not require a long formation. It
is provided with an extensive documentation, available in both english and
french. This documentation has been structured so that the head-teacher,
teachers and students can find the information they are looking for. There are
as many documentation cases as these three situations, meaning we have a
manual for administrators (which should be the head of a course), graders
(i.e. teachers) and students. 

In our institutions, there are one or two teachers that play the role of "referees" for the software, meaning that every encountered issue can be forwarded to them. Thanks to their knowledge about MarkUs, they can help their fellow teachers with problems that can be overcome. When the issue is an unknown one, they can contact the development team, that may open a ticket. It is fundamental to have these people as interfaces with the development team, allowing to synthesize feedback and give hints about awaited features. 

%%%%%%%%%%%%%%%%%%%%%%%%%%%%%%%%%%%%%%%%%%%%%%%%%%%%%%%%%%%%%%%%%%%%%%
\section*{THE IMPACT OF MARKUS}

How the learning process gets improved? How can it be measured? (facts, like: number of students submitting after the deadline; comparisons between the time taken by teachers to comment students works - we have this kind of info from the use if MarkUs by our teachers here at Centrale Nantes, ?)

\subsection*{Analysis criterias}

We evaluate the benefit of Markus according to several criteria, that are following: 
\begin{itemize}
\item Quantitatively: how much does MarkUs increases the speed of the assessment process, how it enhances the number of individual comments left on each students' paper, how students' final evaluation improves thanks to the use of MarkUs during courses;
\item Qualitatively: through annual surveys of teachers and students to validate the adequacy of the tool to teachers' needs, the improvement of teaching and the efficiency in the assessment process.
\end{itemize}
These factors have reinforced the idea that Markus was responding to needs that have been expressed for many years. It naturally fits into the learning context, it is easy to handle and, finally, it improves the understanding of students w.r.t. the reports and works they have to prepare (ability to get a personal comment on the submitted work, respect of deadlines, etc.).

\subsection*{Results}

Markus is a tool that makes life easier for students and graders thanks to a wide range of useful features, such as: the centralized and versioned submission of the papers, its assessment, the annotation of the code and images/PDF documents directly from the web interface, the possibility to download the documents on his own computer to test the code of the students, etc.. As all these  features have been developed specifically for the educational context, MarkUs improves the global pedagogical quality: it offers the possibility to share annotations among multiple teachers and incorporates a correction (and notation) grid defined by the head-supervisor. It also provides additional organizational element: it is easy and quick to make corrections on large cohorts (e.g. by using "reference annotations" that just copy / paste the most recurrent mistakes) and it is quite motivating for teachers. 

\subsection*{Analysis of the impact}

Students have become much more respectful of deadlines than they were before, thanks to the fact of not being able to submit the papers once the due date is past. MarkUs has raised more interest to the annotations left on the platform. In addition, students have enjoued the increased speed correction from their teachers. 

As far as the educational team is concerned, Markus has helped to unify the criteria for marking groups: even if a large number of students do not have the same teachers, their work is assessed according to the same criteria grid. 

\subsection*{Dissemination}

Markus is developed through a strong international collaboration between three institutions of higher education. These institutions include research laboratories. The philosophy of the research is rooted in the DNA of the software. Markus follows a development process perfectly framed, reinforced by rigorous and efficient quality assurance methods. Moreover, we spend much effort to disseminate our results through publications and conferences. The software, its functionality and its impact on the landscape of higher education were presented at various events. MarkUs received the special mention prize at the third "Troph\'{e}es des Technologies Educative" of the french "Salon de l'Education / Educatice". 
The availability of source code from Markus on the collaborative and social platform GitHub gives an international visibility to the project. In addition, we currently help current teachers from various european institutions in their prepratory work  for deploying Markus.


%%%%%%%%%%%%%%%%%%%%%%%%%%%%%%%%%%%%%%%%%%%%%%%%%%%%%%%%%%%%%%%%%%%%%%
\section*{CONCLUSION AND FURTHER WORK}

So far, the students' works were either paper-printed or emailed to their teachers. MarkUs greatly simplifies these steps by centralizing all these documents. It manages classes and and allows students to work in groups. The software allows teachers to get an overview of the different steps achieved by students thanks to versioning features. As the papers and the corresponding personal assessment are centralized on a single platform, every student can access at any time to their work and the related comments, notes and evaluations of their teachers. This is a major breakthrough compared to the traditional use of paper. The integration of MarkUs in the learning environment results in numerous advantages, for both students and teachers.

Current work focus on the implementation of a testing framework into MarkUs. This means the code submitted by students would be immediately compiled and checked w.r.t. criterias defined by teachers. We also investigate the integration of a plagiarism detection tool into the application, to get automatic hints about the similarity of a submission compared to a previous one. Some of these results would be displayed to students, which would also be a way to prevent them from plagiarism. Finally, we currently study how MarkUs could be extended to meet the needs of research, especially during the peer-reviewing process. 

Based on a policy of innovation both demanding (through the quality assurance process that made the quality of the software) and ambitious, Markus owes its success to the close answer it gives to teachers and students' needs. 

%%%%%%%%%%%%%%%%%%%%%%%%%%%%%%%%%%%%%%%%%%%%%%%%%%%%%%%%%%%%%%%%%%%%%%
%\section*{VARIOUS GUIDELINES}

%This article illustrates preparation of ASME paper using \LaTeX2\raisebox{-.3ex}{$\epsilon$}. The \LaTeX\  macro \verb+asme2e.cls+, the {\sc Bib}\TeX\ style file \verb+asmems4.bst+, and the template \verb+asme2e.tex+ that create this article are available on the WWW  at the URL address \verb+http://iel.ucdavis.edu/code/+. To ensure compliance with the 2003 ASME MS4 style guidelines  \cite{asmemanual}, you should modify neither the \LaTeX\ macro \verb+asme2e.cls+ nor the {\sc Bib}\TeX\ style file \verb+asmems4.bst+. By comparing the output generated by typesetting this file and the \LaTeX2\raisebox{-.3ex}{$\epsilon$} source file, you should find everything you need to help you through the preparation of ASME paper using \LaTeX2\raisebox{-.3ex}{$\epsilon$}. Details on using \LaTeX\ can be found in \cite{latex}. Instructions for submitting an electronic version of a paper via ftp for publication on CD-ROM or online  are given at the URL address \verb+http://www.asme.org/pubs/submittal.html+.

%
%An ASME paper should use SI units.  When preference is given to SI units, the U.S. customary units may be given in parentheses or omitted. When U.S. customary units are given preference, the SI equivalent {\em shall} be provided in parentheses or in a supplementary table. 
%%%%%%%%%%%%%%%%%%%%%%%%%%%%%%%%%%%%%%%%%%%%%%%%%%%%%%%%%%%%%%%%%%%%%%%
%\section*{MATHEMATICS}

%Equations should be numbered consecutively beginning with (1) to the end of the paper, including any appendices.  The number should be enclosed in parentheses and set flush right in the column on the same line as the equation.  An extra line of space should be left above and below a displayed equation or formula. \LaTeX\ can automatically keep track of equation numbers in the paper and format almost any equation imaginable. An example is shown in Eqn.~(\ref{eq_ASME}). The number of a referenced equation in the text should be preceded by Eqn.\ unless the reference starts a sentence in which case Eqn.\ should be expanded to Equation.

%\begin{equation}
%f(t) = \int_{0_+}^t F(t) dt + \frac{d g(t)}{d t}
%\label{eq_ASME}
%\end{equation}

%%%%%%%%%%%%%%%%%%%%%%%%%%%%%%%%%%%%%%%%%%%%%%%%%%%%%%%%%%%%%%%%%%%%%%%
%\section*{FIGURES AND TABLES}

%All figures should be positioned at the top of the page where possible.  All figures should be numbered consecutively and captioned; the caption uses all capital letters, and centered under the figure as shown in Fig.~\ref{figure_ASME}. All text within the figure should be no smaller than 7~pt. There should be a minimum two line spaces between figures and text. The number of a referenced figure or table in the text should be preceded by Fig.\ or Tab.\ respectively unless the reference starts a sentence in which case Fig.\ or Tab.\ should be expanded to Figure or Table.

%
%%%%%%%%%%%%%%%%%%%%%%%%%%%%%%%%%%%%%%%%%%%%%%%%%%%%%%%%%%%%%%%%%%%%%%%
%%%%%%%%%%%%%%%%% begin figure %%%%%%%%%%%%%%%%%%%
%\begin{figure}[t]
%\begin{center}
%\setlength{\unitlength}{0.012500in}%
%\begin{picture}(115,35)(255,545)
%\thicklines
%\put(255,545){\framebox(115,35){}}
%\put(275,560){Beautiful Figure}
%\end{picture}
%\end{center}
%\caption{THE FIGURE CAPTION USES CAPITAL LETTERS.}
%\label{figure_ASME} 
%\end{figure}
%%%%%%%%%%%%%%%%% end figure %%%%%%%%%%%%%%%%%%% 
%%%%%%%%%%%%%%%%%%%%%%%%%%%%%%%%%%%%%%%%%%%%%%%%%%%%%%%%%%%%%%%%%%%%%%%

%
%%%%%%%%%%%%%%%%%%%%%%%%%%%%%%%%%%%%%%%%%%%%%%%%%%%%%%%%%%%%%%%%%%%%%%%
%%%%%%%%%%%%%%%% begin table   %%%%%%%%%%%%%%%%%%%%%%%%%%
%\begin{table}[t]
%\caption{THE TABLE CAPTION USES CAPITAL LETTERS, TOO.}
%\begin{center}
%\label{table_ASME}
%\begin{tabular}{c l l}
%& & \\ % put some space after the caption
%\hline
%Example & Time & Cost \\
%\hline
%1 & 12.5 & \$1,000 \\
%2 & 24 & \$2,000 \\
%\hline
%\end{tabular}
%\end{center}
%\end{table}
%%%%%%%%%%%%%%%%% end table %%%%%%%%%%%%%%%%%%% 
%%%%%%%%%%%%%%%%%%%%%%%%%%%%%%%%%%%%%%%%%%%%%%%%%%%%%%%%%%%%%%%%%%%%%%%

%All tables should be numbered consecutively and  captioned; the caption should use all capital letters, and centered above the table as shown in Table~\ref{table_ASME}. The body of the table should be no smaller than 7 pt.  There should be a minimum two line spaces between tables and text.

%%%%%%%%%%%%%%%%%%%%%%%%%%%%%%%%%%%%%%%%%%%%%%%%%%%%%%%%%%%%%%%%%%%%%%%
%\section*{FOOTNOTES\protect\footnotemark}
%\footnotetext{Examine the input file, asme2e.tex, to see how a footnote is given in a head.}

%Footnotes are referenced with superscript numerals and are numbered consecutively from 1 to the end of the paper\footnote{Avoid footnotes if at all possible.}. Footnotes should appear at the bottom of the column in which they are referenced.

%
%%%%%%%%%%%%%%%%%%%%%%%%%%%%%%%%%%%%%%%%%%%%%%%%%%%%%%%%%%%%%%%%%%%%%%%
%\section*{CITING REFERENCES}

%%%%%%%%%%%%%%%%%%%%%%%%%%%%%%%%%%%%%%%%%%%%%%%%%%%%%%%%%%%%%%%%%%%%%%%
%The ASME reference format is defined in the authors kit provided by the ASME.  The format is:

%\begin{quotation}
%{\em Text Citation}. Within the text, references should be cited in  numerical order according to their order of appearance.  The numbered reference citation should be enclosed in brackets.
%\end{quotation}

%The references must appear in the paper in the order that they were cited.  In addition, multiple citations (3 or more in the same brackets) must appear as a `` [1-3]''.  A complete definition of the ASME reference format can be found in the  ASME manual \cite{asmemanual}.

%The bibliography style required by the ASME is unsorted with entries appearing in the order in which the citations appear. If that were the only specification, the standard {\sc Bib}\TeX\ unsrt bibliography style could be used. Unfortunately, the bibliography style required by the ASME has additional requirements (last name followed by first name, periodical volume in boldface, periodical number inside parentheses, etc.) that are not part of the unsrt style. Therefore, to get ASME bibliography formatting, you must use the \verb+asmems4.bst+ bibliography style file with {\sc Bib}\TeX. This file is not part of the standard BibTeX distribution so you'll need to place the file someplace where LaTeX can find it (one possibility is in the same location as the file being typeset).

%With \LaTeX/{\sc Bib}\TeX, \LaTeX\ uses the citation format set by the class file and writes the citation information into the .aux file associated with the \LaTeX\ source. {\sc Bib}\TeX\ reads the .aux file and matches the citations to the entries in the bibliographic data base file specified in the \LaTeX\ source file by the \verb+\bibliography+ command. {\sc Bib}\TeX\ then writes the bibliography in accordance with the rules in the bibliography .bst style file to a .bbl file which \LaTeX\ merges with the source text.  A good description of the use of {\sc Bib}\TeX\ can be found in \cite{latex, goosens} (see how 2 references are handled?).  The following is an example of how three or more references \cite{latex, asmemanual,  goosens} show up using the \verb+asmems4.bst+ bibliography style file in conjunction with the \verb+asme2e.cls+ class file. Here are some more \cite{art, blt, ibk, icn, ips, mts, mis, pro, pts, trt, upd} which can be used to describe almost any sort of reference.

% Here's where you specify the bibliography style file.
% The full file name for the bibliography style file 
% used for an ASME paper is asmems4.bst.
\bibliographystyle{asmems4}


%%%%%%%%%%%%%%%%%%%%%%%%%%%%%%%%%%%%%%%%%%%%%%%%%%%%%%%%%%%%%%%%%%%%%%
\begin{acknowledgment}
Thanks go to D. E. Knuth and L. Lamport for developing the wonderful word processing software packages \TeX\ and \LaTeX. I also would like to thank Ken Sprott, Kirk van Katwyk, and Matt Campbell for fixing bugs in the ASME style file \verb+asme2e.cls+, and Geoff Shiflett for creating 
ASME bibliography stype file \verb+asmems4.bst+.
\end{acknowledgment}

%%%%%%%%%%%%%%%%%%%%%%%%%%%%%%%%%%%%%%%%%%%%%%%%%%%%%%%%%%%%%%%%%%%%%%
% The bibliography is stored in an external database file
% in the BibTeX format (file_name.bib).  The bibliography is
% created by the following command and it will appear in this
% position in the document. You may, of course, create your
% own bibliography by using thebibliography environment as in
%
% \begin{thebibliography}{12}
% ...
% \bibitem{itemreference} D. E. Knudsen.
% {\em 1966 World Bnus Almanac.}
% {Permafrost Press, Novosibirsk.}
% ...
% \end{thebibliography}

% Here's where you specify the bibliography database file.
% The full file name of the bibliography database for this
% article is asme2e.bib. The name for your database is up
% to you.
\bibliography{markus-esda-2012.bib}

%%%%%%%%%%%%%%%%%%%%%%%%%%%%%%%%%%%%%%%%%%%%%%%%%%%%%%%%%%%%%%%%%%%%%%
%\appendix       %%% starting appendix
%\section*{Appendix A: Head of First Appendix}
%Avoid Appendices if possible.

%%%%%%%%%%%%%%%%%%%%%%%%%%%%%%%%%%%%%%%%%%%%%%%%%%%%%%%%%%%%%%%%%%%%%%%
%\section*{Appendix B: Head of Second Appendix}
%\subsection*{Subsection head in appendix}
%The equation counter is not reset in an appendix and the numbers will
%follow one continual sequence from the beginning of the article to the very end as shown in the following example.
%\begin{equation}
%a = b + c.
%\end{equation}

\end{document}
